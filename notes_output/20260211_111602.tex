\documentclass{article}
\usepackage[utf8]{inputenc}
\usepackage{amsmath, amssymb, amsfonts}
\begin{document}
\title{Grenzwert einer Funktion}
\maketitle
\section*{Grenzwert einer Funktion}

Es seien $X$ und $Y$ metrische R\"aume. $D \subset X$ und $f: D \to Y$, weiters sei $a$ ein H\"aufungspunkt von $D$.
Wir schreiben:
$$\lim_{x \to a} f(x) = y$$

falls es ein $y \in Y$ gibt so da\ss f\"ur jede Folge $(x_n)$ in $D$ mit $x_n \to a$ die Folge $(f(x_n))$ in $Y$ gegen $y$ konvergiert.
$$\exists y \in Y: \forall (x_n) \to a \text{ in } D: f(x_n) \to y$$

\textbf{Lemma:} Folgende Aussagen sind \"aquivalent:
\begin{enumerate}
    \item $\lim_{x \to a} f(x) = y$
    \item $\forall V_y \subset Y: \exists U_a \subset X: f(U_a \cap D) \subset V_y$
\end{enumerate}

\textbf{Beweis:} 

$i \Rightarrow ii)$: Durch Kontraposition. Angenommen es gibt eine $V_y$ mit $f(U_a \cap D) \not\subset V_y$ f\"ur jede Umgebung $U_a$ in $X$. Dies gilt dann auch f\"ur $U_a = B_X(a, \frac{1}{n})$ und alle $n \in \mathbb{N}^*$, w\"ahlt man f\"ur gro\"ses $n \in \mathbb{N}^*$ ein $a_n \in B_X(a, \frac{1}{n})$ so gilt $a_n \to a$. Jedoch $f(a_n) \notin V_y$ und

$ii \Rightarrow i)$: F\"ur eine Folge $(x_n) \subset D$ mit $x_n \to a$ und $V_y$ existiert ein $N \in \mathbb{N}$, so dass alle $a_n \in U_a \cap D$ f\"ur $n \ge N$. Wegen $f(U_a \cap D) \subset V_y$ folgt daraus $f(a_n) \in V_y$ f\"ur $n \ge N$ und da $V_y$ beliebig gew\"ahlt war folgt daraus $f(a_n) \to y$, und somit $\lim_{x \to a} f(x) = y$.
\n\section*{Recommended Reading}\n\begin{itemize}\n  \item \textbf{Grundkurs Analysis 2} by Klaus Fritzsche (Match: 0.66)\n  \item \textbf{Lesebuch Mathematik für das erste Studienjahr} by Joachim Hilgert (Match: 0.64)\n  \item \textbf{Lecture Notes} by Topologie (2018) - Andreas Cap (Match: 0.64)\n\end{itemize}\n
\end{document}