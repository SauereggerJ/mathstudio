\documentclass{article}
\usepackage{amsmath, amssymb}
\begin{document}

\begin{center}
\textbf{Series 2}
\end{center}

\textbf{Proposition:} Every absolutely convergent series converges.

\textbf{Proof:} Let $\sum x_k$ be an absolutely convergent series in $E$. Then $\sum |x_k|$ converges in $\mathbb{R}$. Therefore $\sum |x_k|$ satisfies the Cauchy criterion, that is, for all $\varepsilon > 0$ there is some $N$ such that
$$\sum_{k=n+1}^{m} |x_k| < \varepsilon \quad \forall m > n \geq N$$
Since
$$\left| \sum_{k=n+1}^{m} x_k \right| \leq \sum_{k=n+1}^{m} |x_k| < \varepsilon \quad \forall m > n \geq N$$
The series $\sum x_k$ also satisfies the Cauchy criterion.

\textbf{Def:} Let $\sum x_k$ be a series in $E$ and $\sum a_k$ a series in $\mathbb{R}^+$. Then the series $\sum a_k$ is called a majorant (or numerant) for $\sum x_k$ if there is some $K \in \mathbb{N}$ such that $|x_k| \leq a_k$ (or $a_k \leq |x_k|$) for all $k \geq K$.

\textbf{Theorem:} (Majorant criterion) If a series in a Banach space has a convergent majorant, then it converges absolutely.

\textbf{Proof:} Let $\sum x_k$ be a series in $E$ and $\sum a_k$ a convergent majorant. Then there is some $K$ such that $|x_k| \leq a_k$ for all $k \geq K$. By Cauchy criterion for $\varepsilon > 0$ there is some $N \geq K$ such that
$$\sum_{k=n+1}^{m} a_k < \varepsilon \quad \text{for all } m > n \geq N$$
So we have:
$$\left| \sum_{k=n+1}^{m} x_k \right| \leq \sum_{k=n+1}^{m} |x_k| \leq \sum_{k=n+1}^{m} a_k < \varepsilon \quad m > n \geq N$$
Therefore $\sum |x_k|$ converges and $\sum x_k$ converges absolutely.

\n\section*{Recommended Reading}\n\begin{itemize}\n  \item \textbf{Analysis in Banach Spaces (Vol III)} by Hytoenen et al (Match: 0.69)\n  \item \textbf{Analysis in Banach Spaces : Volume II} by Tuomas Hytönen, Jan van Neerven, Mark Veraar, Lutz Weis (Match: 0.69)\n  \item \textbf{Analysis I} by H. Amann (Match: 0.68)\n\end{itemize}\n
\end{document}