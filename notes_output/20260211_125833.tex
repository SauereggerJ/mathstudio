\documentclass{article}
\usepackage{amsmath}
\usepackage{amssymb}
\title{Boundary of a Set in $(M,d)$}
\author{}
\date{}
\begin{document}
\maketitle

\section*{Boundary of a Set in $(M,d)$}

\noindent\textbf{Definition:} For a given set $A \subset M$, the boundary of $A$ is
$$\partial A = \overline{A} \cap \overline{\left(A^c\right)}$$
(Interpreting $\text{cl}(A)$ as $\overline{A}$)

\vspace{0.5cm}
\noindent\textbf{Examples:}
$$\partial M = \emptyset, \quad \partial \mathbb{Z} = \emptyset, \quad \partial \mathbb{Q} = \mathbb{R}$$

\vspace{0.5cm}
\noindent\textbf{Lemma 1:} $x \in \partial A \iff \forall \varepsilon > 0, B_{\varepsilon}(x) \cap A \neq \emptyset \text{ and } B_{\varepsilon}(x) \cap A^c \neq \emptyset$.

\noindent\textbf{Proof (for $\Rightarrow$):}
$$x \in \partial A \iff x \in \overline{A} \text{ and } x \in \overline{\left(A^c\right)}$$
Since $x \in \overline{A}$ means that for every $\varepsilon > 0$, $B_{\varepsilon}(x) \cap A \neq \emptyset$, and $x \in \overline{\left(A^c\right)}$ means that for every $\varepsilon > 0$, $B_{\varepsilon}(x) \cap A^c \neq \emptyset$.

\vspace{0.5cm}
\noindent\textbf{Lemma 2:} $\partial A = \overline{A} \setminus \text{int}(A) = \overline{A} \cap (\text{int}(A))^c$.

\noindent\textbf{Proof (for $\Leftarrow$ direction of showing $\overline{A} \cap (\text{int}(A))^c$):}
We use the fact that $(\text{int}(A))^c = \overline{\left(A^c\right)}$ because
$$x \in (\text{int}(A))^c \iff x \notin \text{int}(A)$$
$$\iff \forall \varepsilon > 0: B_{\varepsilon}(x) \cap A^c \neq \emptyset$$
$$\iff x \in \overline{A^c}$$

\n\section*{Recommended Reading}\n\begin{itemize}\n  \item \textbf{Writing Proofs in Analysis} by Jonathan M. Kane (Match: 0.67)\n  \item \textbf{Measure Theory and Fine Properties of Functions} by Gariepy & Evans (Match: 0.66)\n  \item \textbf{Real Analysis} by Patrick Fitzpatrick (Match: 0.66)\n\end{itemize}\n
\end{document}