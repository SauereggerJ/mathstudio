\documentclass{article}
\usepackage{amsmath}
\usepackage{amssymb}
\begin{document}

\section*{Problem 1}
Find a list of four distinct vectors in $\mathbb{F}^3$ whose span equals $V = \{(x, y, z) \in \mathbb{F}^3 : x+y+z=0\}$.

\textbf{Solution:}
For every $\mathbf{v} \in V$ holds: $x+y=-z$, and so every vector of $V$ must be of the form $\mathbf{v} = (x, y, -x-y)$. Such vectors can be constructed by a linear combination of the two vectors $\mathbf{v}_1 = (1, 0, -1)$ and $\mathbf{v}_2 = (0, 1, -1)$ because of
$$\mathbf{v} = x \cdot \mathbf{v}_1 + y \cdot \mathbf{v}_2.$$
So $V \subseteq \text{span}(\mathbf{v}_1, \mathbf{v}_2)$.
On the other side, it holds that every linear combination of $\mathbf{v}_1$ and $\mathbf{v}_2$ fulfills $x+y+z=0$, and so $\text{span}(\mathbf{v}_1, \mathbf{v}_2) \subseteq V$.
Therefore $V = \text{span}(\mathbf{v}_1, \mathbf{v}_2)$.
To get the solution just add two linear combinations of $\mathbf{v}_1$ and $\mathbf{v}_2$, say $\mathbf{v}_3 = \mathbf{v}_1 + \mathbf{v}_2$ and $\mathbf{v}_4 = \mathbf{v}_1 - \mathbf{v}_2$.

\section*{Problem 2}
Prove or give a counterexample: If $\mathbf{v}_1, \mathbf{v}_2, \mathbf{v}_3, \mathbf{v}_4$ spans $V$, then the list $\mathbf{v}_1 - \mathbf{v}_2, \mathbf{v}_2 - \mathbf{v}_3, \mathbf{v}_3 - \mathbf{v}_4, \mathbf{v}_4$ also spans $V$.

\textbf{Solution:}
The list also spans $V$ because:
$$\mathbf{v}_4 = \mathbf{v}_4$$
$$\mathbf{v}_3 = (\mathbf{v}_3 - \mathbf{v}_4) + \mathbf{v}_4$$
$$\mathbf{v}_2 = (\mathbf{v}_2 - \mathbf{v}_3) + (\mathbf{v}_3 - \mathbf{v}_4) + \mathbf{v}_4$$
$$\mathbf{v}_1 = (\mathbf{v}_1 - \mathbf{v}_2) + (\mathbf{v}_2 - \mathbf{v}_3) + (\mathbf{v}_3 - \mathbf{v}_4) + \mathbf{v}_4$$
Therefore every linear combination of $\mathbf{v}_1, \mathbf{v}_2, \mathbf{v}_3, \mathbf{v}_4$ can also be written in a linear combination of the new list.

\n\section*{Recommended Reading}\n\begin{itemize}\n  \item \textbf{Prüfungstraining Lineare Algebra : Band I} by Thomas C. T. Michaels (Match: 0.71)\n  \item \textbf{Linear algebra problem book} by Paul R. Halmos (Match: 0.70)\n  \item \textbf{Lineare Algebra 1} by Menny-Akka (Match: 0.70)\n\end{itemize}\n
\end{document}