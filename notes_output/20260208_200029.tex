\documentclass{article}
\usepackage[utf8]{inputenc}
\usepackage{amsmath}
\usepackage{amsfonts}
\usepackage{amssymb}

\begin{document}

\section*{\centering \color[HTML]{2E7D32} Matrix Representation eines Linear Operators}

Suppose $T \in \mathcal{L}(V)$. The matrix of $T$ with respect to a basis $v_1, \dots, v_n$ of $V$ is the $n \times n$ matrix
\[
\mathcal{M}(T) = \begin{pmatrix}
A_{1,1} & \dots & A_{1,n} \\
\vdots & & \vdots \\
A_{n,1} & \dots & A_{n,n}
\end{pmatrix}
\]
whose entries $A_{j,k}$ are defined by: $T v_k = A_{1,k} v_1 + A_{2,k} v_2 + \dots + A_{n,k} v_n$

{\color{red} The $k^{\text{th}}$ column of the matrix $\mathcal{M}(T)$ is formed from the coefficients used to write $T v_k$ as a linear combination of the basis $v_1, \dots, v_n$}

Examples: $T \in \mathcal{L}(\mathbb{F}^3)$ by $T(x, y, z) = (2x + y, 5y + 3z, 8z)$, then $\mathcal{M}(T)$ with respect to the standard basis of $\mathbb{F}^3$ is
\[
\mathcal{M}(T) = \begin{pmatrix}
2 & 1 & 0 \\
0 & 5 & 3 \\
0 & 0 & 8
\end{pmatrix}
\]

\n\section*{Recommended Reading}\n\begin{itemize}\n  \item \textbf{Lineare Algebra 1} by Menny-Akka (Match: 0.70)\n  \item \textbf{Prüfungstraining Lineare Algebra : Band I} by Thomas C. T. Michaels (Match: 0.70)\n  \item \textbf{Tutorium Analysis 1 und Lineare Algebra 1} by Florian Modler, Martin Kreh (Match: 0.69)\n\end{itemize}\n
\end{document}