\documentclass{article}
\usepackage{amsmath, amssymb}
\title{Connectedness}
\author{}
\date{}
\begin{document}
\maketitle

\section*{Connectedness}

\noindent\textbf{Definition.}: If $X$ is a topological space, a separation of $X$ is a pair of nonempty, disjoint open subsets $U, V \subset X$ such that $U \cup V = X$.

$X$ is disconnected if there exists a separation of $X$, and connected otherwise.

\section*{Examples:}

\begin{enumerate}
    \item $X$ is the union of the two disjoint closed disks $\overline{B_{1}(2,0)}$ and $\overline{B_{1}(-2,0)}$ in $\mathbb{R}^2$. Each of the disks is open in $X$, so the pair of disks is a separation of $X$.
    \item $Y = \mathbb{R} \setminus \{0\}$, the two sets $(-\infty, 0)$ and $(0, \infty)$ separate $Y$.
    ($(-\infty, 0)$ is short for $\{x \in Y \mid x < 0\}$.)
    \item $Z = $ the set of points with rational coordinates. A separation is given by $U = \{(x,y) \mid x < \pi\}$ and $V = \{(x,y) \mid x > \pi\}$.
\end{enumerate}
\n\section*{Recommended Reading}\n\begin{itemize}\n  \item \textbf{Topology} by James Munkres (Match: 0.69)\n  \item \textbf{Lecture Notes} by Topologie (2018) - Andreas Cap (Match: 0.68)\n  \item \textbf{Lecture Notes} by Topologie - Andreas Kriegl (Match: 0.68)\n\end{itemize}\n
\end{document}