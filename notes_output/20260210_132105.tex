\documentclass{article}
\usepackage{amsmath}

\begin{document}

\textbf{Subspaces, Sum of Subspaces and Direct Sums}

\textbf{Definition:} A subset $M$ of $V$ is called a subspace of $V$ if $M$ is also a vector space with the same additive identity, addition, and scalar multiplication as on $V$.

\textbf{Definition:} Suppose $V_1, \dots, V_m$ are subspaces of $V$. The sum of $V_1, \dots, V_m$, denoted by $V_1 + \dots + V_m$ is the set of all possible sums of elements of $V_1, \dots, V_m$. More precisely,

$$V_1 + \dots + V_m = \{v_1 + \dots + v_m : v_1 \in V_1, \dots, v_m \in V_m\}$$

\textbf{Definition:} Suppose $V_1, \dots, V_m$ are subspaces of $V$.
\begin{itemize}
    \item The sum $V_1 + \dots + V_m$ is called a direct sum if each element of $V_1 + \dots + V_m$ can be written in only one way as a sum $v_1 + \dots + v_m$, where each $v_k \in V_k$.
    \item If $V_1 + \dots + V_m$ is a direct sum, then $V_1 \oplus \dots \oplus V_m$ denotes $V_1 + \dots + V_m$, with the $\oplus$ notation serving as an indication that this is a direct sum.
\end{itemize}

\n\section*{Recommended Reading}\n\begin{itemize}\n  \item \textbf{Lineare Algebra 1} by Menny-Akka (Match: 0.70)\n  \item \textbf{Groups, Matrices, and Vector Spaces} by James B. Carrell (Match: 0.69)\n  \item \textbf{Prüfungstraining Lineare Algebra : Band I} by Thomas C. T. Michaels (Match: 0.69)\n\end{itemize}\n
\end{document}