\documentclass{article}
\usepackage{amsmath}
\usepackage{amssymb}
\begin{document}

Let $X$ and $Y$ be normed spaces. A mapping $f: E \to Y$ of a set $E \subset X$ into $Y$ is differentiable at an interior point $x \in E$ if there exists a continuous linear transformation $\mathcal{L}(x): X \to Y$ such that
\[
f(x+h) - f(x) = \mathcal{L}(x)\cdot h + \alpha(x;h) \quad (*)
\]
where $\alpha(x,h) = o(\lVert h \rVert)$ as $h \to 0$, $x+h \in E$.

\textbf{Proposition:} If a mapping $f: E \to Y$ is differentiable at an interior point $x$ of a set $E \subset X$, its differential $\mathcal{L}(x)$ at that point is uniquely determined.

\textbf{Proof:} Let $\mathcal{L}_1(x)$ and $\mathcal{L}_2(x)$ be linear mappings satisfying $(*)$, that is
\[
f(x+h) - f(x) = \mathcal{L}_1(x)\cdot h + \alpha_1(x,h)
f(x+h) - f(x) = \mathcal{L}_2(x)\cdot h + \alpha_2(x,h)
\]
where $\alpha_i(x;h) = o(\lVert h \rVert)$ as $h \to 0$, $x+h \in E$, $i=1, 2$.

Then $\mathcal{L}(x) = \mathcal{L}_2(x) - \mathcal{L}_1(x)$ and $\alpha(x;h) = \alpha_2(x;h) - \alpha_1(x;h)$ we obtain
\[
\mathcal{L}(x)h = \alpha(x;h)
\]

Here $\mathcal{L}(x)$ is a mapping that is linear with respect to $h$, and $\alpha(x;h) = o(\lVert h \rVert)$ as $h \to 0$, $x+h \in E$.

Taking an auxiliary numerical parameter $\lambda$, we can now write
\[
\lVert \mathcal{L}(x)h \rVert = \frac{\lVert \mathcal{L}(x)(\lambda h) \rVert}{\|\lambda\|} = \frac{\lVert \alpha(x;\lambda h) \rVert}{\|\lambda h\rVert} \frac{\|h\rVert}{\|\lambda\|} = \frac{\lVert \alpha(x;\lambda h) \rVert}{\|\lambda h\rVert} \lVert h \rVert
\]
Thus $\mathcal{L}(x)h = 0$ for any $h \ne 0$.

\n\section*{Recommended Reading}\n\begin{itemize}\n  \item \textbf{Linear Functional Analysis} by Alt & Nuernberg (Match: 0.71)\n  \item \textbf{Fréchet differentiability of Lipschitz functions and porous sets in Banach spaces} by Joram Lindenstrauss (Match: 0.70)\n  \item \textbf{Linear Functional Analysis} by Hans Wilhelm Alt (Match: 0.70)\n\end{itemize}\n
\end{document}