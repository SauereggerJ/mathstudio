\documentclass{article}
\usepackage{amsmath, amssymb, amsthm}
\usepackage[ngerman]{babel}
\theoremstyle{definition}
\newtheorem*{theorem}{Satz}
\newtheorem*{proof}{Beweis}
\newtheorem*{remark}{Bemerkung}
\begin{document}

\begin{center}
\Large\textbf{Mittelwertsatz f\"ur vektorwertige Funktionen}
\end{center}

Es sei $E$ ein normierter Vektorraum und $f \in C([a, b], E)$ sowie in $(a, b)$ differenzierbar.
Dann gilt
$$
\|f(b) - f(a)\| \le \sup_{t \in (a, b)} \|f'(t)\| (b-a)
$$

\begin{proof} Falls $f'$ unbeschr{"a}nkt ist, bleibt nichts mehr zu zeigen, sei also $f'$ beschr{"a}nkt und $\alpha > 0$ mit $\alpha > \|f'(t)\| \quad \forall t \in (a, b)$. Sei $\varepsilon \in (0, b-a)$ f{"u}r und
$$S := \{ \sigma \in [a+\varepsilon, b] \mid \|f(\sigma) - f(a)\| \le \alpha (\sigma - a) \},$$
Sist also die Menge aller Punkte zwischen $a+\varepsilon$ und $b$ f{"u}r die der Satz gilt.

1) Die Menge $S$ ist nicht leer, da $a+\varepsilon \in S$ geh{"o}rt.

2) $S$ ist abgeschlossen, $\tilde{\rho}(\sigma) = \frac{\|f(\sigma) - f(a)\|}{\sigma - a}$ ist stetig und $S = \tilde{\rho}^{-1}([0; \alpha(\sigma - a - \varepsilon)])$ 
oder
$$p(\sigma) = \frac{\|f(\sigma) - f(a+\varepsilon)\|}{\sigma - (a+\varepsilon)} \quad \text{und} \quad \tilde{S} = p^{-1}([0; \alpha])$$ 
und
$$S = \{a+\varepsilon\} \cup \tilde{S} = \{ \sigma \in [a+\varepsilon, b] \mid \|f(\sigma) - f(a+\varepsilon)\| \le \alpha(\sigma - (a+\varepsilon)) \}$$ 
sein so: $p(a+\varepsilon) = \|f'(a+\varepsilon)\| \le \alpha$ und $S = p^{-1}((-\infty, \alpha])$

3) also ist $S$ kompakt, und es existiert $s = \max S$.

4) Sei nun $s < b$. Dann gilt f{"u}r $t \in (s, b)$ 
$$
\|f(t) - f(a+\varepsilon)\| = \|f(t) - f(s) + f(s) - f(a+\varepsilon)\| \\
\le \|f(t) - f(s)\| + \|f(s) - f(a+\varepsilon)\| \le \|f(t) - f(s)\| + \alpha \cdot (s - (a+\varepsilon)) 
$$
Da $f$ auf $[a+\varepsilon, b]$ differenzierbar ist folgt
$$
\frac{\|f(t) - f(s)\|}{t - s} \to \|f'(s)\| \quad (t \to s).
$$
Dies bedeutet aber f{"u}r $t$ nach genug bei $s$, oder $\exists \delta \in (0, b-s)$ mit
$$ \frac{\|f(t) - f(s)\|}{t - s} \le \alpha \cdot (t - s), \quad 0 < t - s < \delta $$

Damit folgt:
$$ \|f(t) - f(a+\varepsilon)\| \le \alpha \cdot (t - a - \varepsilon) \quad s < t < s+\delta $$
im Widerspruch zur Maximalit{"a}t von $S$, also gilt $s=b$.

5) Somit gilt $\|f(b) - f(a+\varepsilon)\| \le \alpha \cdot (b-a-\varepsilon)$ f{"u}r jede obere Schranke $\alpha$ 
von $\{\|f'(t)\|; t \in (a, b)\}$, also $\|f(b) - f(a+\varepsilon)\| \le \sup_{t \in (a, b)} \|f'(t)\| (b-a-\varepsilon)$

Hund da dies f{"u}r jedes $\varepsilon \in (0, b-a)$ richtig ist folgt die Behauptung durch den Gren{"u}bergang $\varepsilon \to 0$ wegen der Stetigkeit von $f$.

\end{proof}
\n\section*{Recommended Reading}\n\begin{itemize}\n  \item \textbf{Lineare Funktionalanalysis Eine Anwendungsorientierte Einfhrung} by Hans Wilhelm Alt (Match: 0.70)\n  \item \textbf{Grundkurs Analysis 2} by Klaus Fritzsche (Match: 0.69)\n  \item \textbf{Angewandte Funktionalanalysis} by Manfred Dobrowolski (Match: 0.69)\n\end{itemize}\n
\end{document}