\documentclass{article}
\usepackage{amsmath}
\usepackage{amssymb}
\begin{document}

\section*{Topological Space}

A topological space is a set $X$, and a family $\mathcal{T}$ of subsets of $X$ (the topology) that obey three axioms:
\begin{enumerate}
    \item $\emptyset, X \in \mathcal{T}$.
    \item If $A_1, \ldots, A_n \in \mathcal{T}$, then $A_1 \cap \cdots \cap A_n \in \mathcal{T}$.
    \item If $\{A_\alpha\}_{\alpha \in I} \subset \mathcal{T}$, then $\bigcup_{\alpha \in I} A_\alpha \in \mathcal{T}$.
\end{enumerate}

\section*{Base and Subbase}

\begin{itemize}
    \item A base of a topology $\mathcal{T}$ is a subset $\mathcal{B}$ of $\mathcal{T}$ so that any element of $\mathcal{T}$ is a union of sets of $\mathcal{B}$.
    \item A subbase of a topology is a subset $\mathcal{S}$, so that the family of finite intersections of sets in $\mathcal{S}$ is a base.
\end{itemize}

This definition is from an analytical point of view.

The topology already defined and $\mathcal{B}$ is a subset of $\mathcal{T}$.

\begin{itemize}
    \item Suppose $X$ is any set. A basis in $X$ is a collection $\mathcal{B}$ of subsets of $X$ satisfying the following conditions:
    \begin{enumerate}
        \item Every element of $X$ is in some elements of $\mathcal{B}$, in other words, $X = \bigcup_{B \in \mathcal{B}} B$.
        \item If $B_1, B_2 \in \mathcal{B}$ and $x \in B_1 \cap B_2$, there exists an element $B_3 \in \mathcal{B}$ such that $x \in B_3 \subset B_1 \cap B_2$.
    \end{enumerate}
\end{itemize}

This constructs a topology, as proven in the next proposition.

\n\section*{Recommended Reading}\n\begin{itemize}\n  \item \textbf{Lecture Notes} by Topologie - Andreas Kriegl (Match: 0.72)\n  \item \textbf{Lecture Notes} by Topologie (2018) - Andreas Cap (Match: 0.71)\n  \item \textbf{Topology} by James Munkres (Match: 0.71)\n\end{itemize}\n
\end{document}