\documentclass{article}
\usepackage{amsmath, amssymb}
\begin{document}

\section*{Math Notes}

\noindent\textbf{Definition:}
Suppose $T \in \mathcal{L}(V, W)$. The dual map of $T$ is the linear map $T^{\prime} \in \mathcal{L}(W^{\prime}, V^{\prime})$ defined for each $\varphi \in W^{\prime}$ by
$$T^{\prime}(\varphi) = \varphi \circ T$$

\noindent\textbf{Two remarks:}
\begin{enumerate}
    \item The image of $T^{\prime}$ is a linear map from $V$ to $\mathbb{F}$ because $\varphi \circ T$ is a composition of two linear maps.
    \item $T^{\prime}$ itself is also linear:
    \begin{itemize}
        \item For $\varphi, \psi \in W^{\prime}$:
        $$T^{\prime}(\varphi + \psi) = (\varphi + \psi) \circ T = \varphi \circ T + \psi \circ T = T^{\prime}(\varphi) + T^{\prime}(\psi)$$
        \item For $\lambda \in \mathbb{F}$ and $\varphi \in W^{\prime}$:
        $$T^{\prime}(\lambda \varphi) = (\lambda \varphi) \circ T = \lambda (\varphi \circ T) = \lambda T^{\prime}(\varphi)$$
    \end{itemize}
\end{enumerate}

\section*{Examples}

\noindent\textbf{Example 1:}
Let $D: \mathcal{P}(\mathbb{R}) \to \mathcal{P}(\mathbb{R})$ be defined by $D(p) = p^{\prime}$ (differentiation).

\noindent 1) Suppose $\varphi$ is the linear functional on $\mathcal{P}(\mathbb{R})$ defined by $\varphi(p) = p(3)$.

\n\section*{Recommended Reading}\n\begin{itemize}\n  \item \textbf{Tutorium Analysis 2 und Lineare Algebra 2} by Florian Modler, Martin Kreh (Match: 0.68)\n  \item \textbf{Dualities and Representations of Lie Superalgebras} by Cheng & Wang (Match: 0.68)\n  \item \textbf{Lineare Funktionalanalysis Eine Anwendungsorientierte Einfhrung} by Hans Wilhelm Alt (Match: 0.67)\n\end{itemize}\n
\end{document}