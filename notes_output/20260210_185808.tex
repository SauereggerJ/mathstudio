\documentclass{article}
\usepackage{amsmath, amssymb, amsthm}
\begin{document}

\section*{The Limit Superior and Limit Inferior}

\begin{definition}[Definition]
Let $(x_n)$ be a sequence in $\mathbb{R}$. We can define two new sequences $(y_n)$ and $(z_n)$ by
$$
y_n := \sup_{k \ge n} x_k := \sup \{x_k ; k \ge n\}$$
$$
z_n := \inf_{k \ge n} x_k := \inf \{x_k ; k \ge n\}$$
\end{definition}

Since $(x_n)$ is decreasing and $(z_n)$ is increasing in $\mathbb{R}$, therefore these sequences converge in $\mathbb{R}$.

The limit superior is defined as:
$$\limsup_{n \to \infty} x_n := \lim_{n \to \infty} y_n := \lim_{n \to \infty} \left( \sup_{k \ge n} x_k \right)$$
The limit inferior is defined as:
$$\liminf_{n \to \infty} x_n := \lim_{n \to \infty} z_n := \lim_{n \to \infty} \left( \inf_{k \ge n} x_k \right)$$

We also have:
$$\limsup_{n \to \infty} x_n = \inf_{n \in \mathbb{N}} y_n = \inf_{n \in \mathbb{N}} \left( \sup_{k \ge n} x_k \right) \quad \text{and} \quad \liminf_{n \to \infty} x_n = \sup_{n \in \mathbb{N}} z_n = \sup_{n \in \mathbb{N}} \left( \inf_{k \ge n} x_k \right)$$

\begin{theorem}
Any sequence $(x_n)$ in $\mathbb{R}$ has a smallest cluster point $x_*$ and a greatest cluster point $x^*$ in $\overline{\mathbb{R}}$ and these satisfy:
$$\liminf_{n \to \infty} x_n = x_* \quad \text{and} \quad \limsup_{n \to \infty} x_n = x^*$$
\end{theorem}

\begin{proof}
Let $x^* = \limsup_{n \to \infty} x_n$ and $y_n := \sup_{k \ge n} x_k$ for $n \in \mathbb{N}$. Then $(y_n)$ is a decreasing sequence such that
$$x^* = \inf_{n \in \mathbb{N}} y_n$$
We consider 3 cases:
\begin{enumerate}
    \item $x^* = -\infty$. Then for each $K > 0$, there is some $n$ such that $-K > y_n = \sup_{k \ge n} x_k$. Otherwise we would have $x_k \ge -K_0$ for some $K_0 \ge 0$. Hence $x_k \in (-\infty, K)$ for all $k \ge n$, that is $x^* = -\infty$ is the only cluster point of $(x_n)$.
    \item Suppose that $x^* > -\infty$. For each $\varepsilon > x^*$, we have some $n$ such that $y_n = \sup_{k \ge n} x_k \ge x^*$. The terms $x_k$ for $k \ge n$ are all smaller than $x^*+\varepsilon$. So no cluster point of $(x_n)$ is larger than $x^*$.
    Since $\sup_{k \ge n} x_k = y_n \ge x^*$ for all $n \in \mathbb{N}$, we have for each $n$, some $k \ge n$ such that $x_k > x^* - \varepsilon$. 
    Since we already know that no cluster point of $(x_n)$ is larger than $x^*$, the interval $(x^* - \varepsilon, x^* + \varepsilon)$ must contain infinitely many terms of $(x_n)$. That is $x^*$ is a cluster point of $(x_n)$.
    \item $x^* = \infty$. Because of $x^* = \inf_{n \in \mathbb{N}} y_n$, we have $y_n = \infty$ for all $n \in \mathbb{N}$. Since for each $K > 0$ and any $n$, there is some $k \ge n$ such that $x_k > K$. This means that $x^* = \infty$ is a cluster point of $(x_n)$, and clearly the largest such point.
\end{enumerate}
The proof that $x_* = \liminf_{n \to \infty} x_n$ is similar.
\end{proof}

\n\section*{Recommended Reading}\n\begin{itemize}\n  \item \textbf{Writing Proofs in Analysis} by Jonathan M. Kane (Match: 0.68)\n  \item \textbf{Problems in real analysis} by Teodora-Liliana Rădulescu (Match: 0.67)\n  \item \textbf{Undergraduate Analysis} by McCluskey & McMaster (Match: 0.65)\n\end{itemize}\n
\end{document}