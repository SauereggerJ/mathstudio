\documentclass{article}
\usepackage[utf8]{inputenc}
\usepackage{amsmath, amssymb, amsfonts}
\begin{document}
\title{Properties of the Dual Basis}
\maketitle
\section*{2.M Axler}

\begin{defn} If $\mathbf{v}_1,\dots,\mathbf{v}_n$ is a basis of $V$, then the dual basis of $\mathbf{v}_1,\dots,\mathbf{v}_n$ is the list $\varphi_1,\dots,\varphi_n$ of elements of $V'$, where each $\varphi_j$ is the linear functional on $V$ such that
$$\varphi_j(\mathbf{v}_k) = \begin{cases} 1 & \text{if } k=j \\ 0 & \text{if } k \neq j \end{cases}$$
\end{defn}

\begin{lemma} Suppose $\mathbf{v}_1,\dots,\mathbf{v}_n$ is a basis of $V$ and $\varphi_1,\dots,\varphi_n$ is the dual basis. Then, for each $\mathbf{v} \in V$, we have
$$\mathbf{v} = \varphi_1(\mathbf{v})\mathbf{v}_1 + \dots + \varphi_n(\mathbf{v})\mathbf{v}_n$$
\end{lemma}

\begin{proof} Suppose $\mathbf{v} \in V$. We write $\mathbf{v}$ in terms of the basis $\{\mathbf{v}_k\}$:
$$\mathbf{v} = c_1\mathbf{v}_1 + \dots + c_n\mathbf{v}_n$$ 
Applying $\varphi_j$ to both sides yields
$$\varphi_j(\mathbf{v}) = \varphi_j(c_1\mathbf{v}_1 + \dots + c_n\mathbf{v}_n) = c_1\varphi_j(\mathbf{v}_1) + \dots + c_n\varphi_j(\mathbf{v}_n) = c_j$$
Thus, $\mathbf{v} = \varphi_1(\mathbf{v})\mathbf{v}_1 + \dots + \varphi_n(\mathbf{v})\mathbf{v}_n$.
\end{proof}

\begin{lemma} Suppose $V$ is finite-dimensional. Then the dual of a basis of $V$ is a basis of $V'$. Let $\varphi_1,\dots,\varphi_n$ denote the dual basis.
\end{lemma}

\begin{proof} Suppose $\mathbf{v}_1,\dots,\mathbf{v}_n$ is a basis of $V$. Let $\varphi_1,\dots,\varphi_n$ denote the dual basis.

First, we show that $\varphi_1,\dots,\varphi_n$ are linearly independent.
Suppose $a_1\varphi_1 + \dots + a_n\varphi_n = 0$ (the zero linear map).

Then, for each $\mathbf{v}_k$:
$$(a_1\varphi_1 + \dots + a_n\varphi_n)(\mathbf{v}_k) = a_1\varphi_1(\mathbf{v}_k) + \dots + a_n\varphi_n(\mathbf{v}_k) = a_k$$ 
Since the linear map is zero, we must have $a_k = 0$ for each $k$.

Because the length of the dual basis is $n$, this linearly independent list must be a basis.
\end{proof}
\n\section*{Recommended Reading}\n\begin{itemize}\n  \item \textbf{Linear Algebra Done Right} by Sheldon Axler (Match: 0.71)\n  \item \textbf{Linear algebra problem book} by Paul R. Halmos (Match: 0.70)\n  \item \textbf{Linear Algebra} by Seymour Lipschutz (Match: 0.70)\n\end{itemize}\n
\end{document}