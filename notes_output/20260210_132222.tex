\documentclass{article}
\usepackage{amsmath}

\begin{document}

\textbf{Theorem:} In a finite-dimensional vector space, the length of every linearly independent list of vectors is less than or equal to the length of every spanning list of vectors. (A, page 35)

\textbf{Theorem:} Every subspace of a finite-dimensional vector space is finite-dimensional. (A, page 36)

\section*{Bases and Dimension}

\textbf{Definition:} A basis is a list of vectors in $V$ that is linearly independent and spans $V$.

\textbf{Theorem:} A list $v_1, \dots, v_n$ of vectors is a basis of $V$ if and only if every $v \in V$ can be written uniquely in the form

$$V = a_1 v_1 + \dots + a_n v_n, \quad a_1, \dots, a_n \in F$$
(A, page 39)

\textbf{Theorem:} Every spanning list in a vector space can be reduced to a basis of the vector space. (A, page 40)

\textbf{Theorem:} Every finite-dimensional vector space has a basis. (A, page 41)

\textbf{Theorem:} Every linearly independent list of vectors in a finite-dimensional vector space can be extended to a basis of the vector space. (A, page 41)

\textbf{Theorem:} Suppose $V$ is finite dimensional and $U$ is a subspace of $V$. Then there is a subspace $W$ of $V$ such that $V = U \oplus W$.

\n\section*{Recommended Reading}\n\begin{itemize}\n  \item \textbf{Lineare Algebra 1} by Menny-Akka (Match: 0.72)\n  \item \textbf{Groups, Matrices, and Vector Spaces} by James B. Carrell (Match: 0.71)\n  \item \textbf{Prüfungstraining Lineare Algebra : Band I} by Thomas C. T. Michaels (Match: 0.71)\n\end{itemize}\n
\end{document}