\documentclass{article}
\usepackage{amsmath}

\begin{document}

\textbf{Definition:} For $T \in \mathcal{L}(V,W)$, the null space of $T$ is the subset of $V$ with $T(v) = 0$.\\
$\text{null } T = \{v \in V \mid T(v) = 0\}$.

\textbf{Definition:} A function $T: V \to W$ is called injective if $T(v) = T(w)$ implies $v=w$, or equivalently $v \neq w \implies T(v) \neq T(w)$.

\textbf{Definition:} For $T \in \mathcal{L}(V,W)$, the range of $T$ is the subset of $W$ consisting of vectors that are equal to $T(v)$ for a $v \in V$.\\
$\text{range } T = \{w \in W \mid Tv = w\}$

\textbf{Definition:} A function $T: V \to W$ is called surjective if $W = \text{range } T$.

\textbf{Theorem:} Suppose $T \in \mathcal{L}(V,W)$, then $\text{null } T$ is a subspace of $V$.\\
\textit{Proof:} Because of $T(0) = 0$, $0$ is an element of $\text{null } T$. For $v, w \in \text{null } T$ we have $T(v+w) = T(v) + T(w) = 0 + 0 = 0$, so $v+w \in \text{null } T$. And with $\lambda \in \mathbb{K}$ we get $T(\lambda w) = \lambda T(w) = \lambda 0 = 0$, so $\lambda v$ is also an element of $\text{null } T$, and $\text{null } T$ therefore is a subspace of $V$.

\textbf{Theorem:} For $T \in \mathcal{L}(V,W)$, $\text{null } T = \{0\}$ if and only if $T$ is injective.\\
\textit{Proof:} Suppose $T$ is injective and $v \in V$ is an element of $\text{null } T$, then $T(v) = 0 = T(0)$ and because of injectivity $v = 0$. Therefore $\text{null } T = \{0\}$.

Suppose now $\text{null } T = \{0\}$ and $T(v) = T(w)$ for two elements of $V$, then $T(v) - T(w) = T(v-w) = 0$. This means $v-w \in \text{null } T$, but so $v-w$ must be $0$, and this yields $v=w$, therefore $T$ is injective.

\textbf{Lemma} (because we used it, twice already): $0 \in \text{null } T$.\\
\textit{Proof:} $T(0) = T(v-v) = T(v) - T(v) = 0$.

\textbf{Theorem:} Suppose $T \in \mathcal{L}(V,W)$, then $\text{range } T$ is a subspace of $W$.\\
\textit{Proof:} 1) $0 \in \text{range } W$ because $T(0) = 0$, if $w_1, w_2 \in \text{range } T$, then there exist $v_1, v_2$ with $T(v_1) = w_1$ and $T(v_2) = w_2$ and so $T(v_1 + v_2) = T(v_1) + T(v_2) = w_1 + w_2$, therefore $w_1 + w_2 \in \text{range } W$, if $w \in \text{range } T$ and $\lambda \in \mathbb{K}$, then there exist a $v \in V$ with $T(v) = w$, and so $T(\lambda v) = \lambda T(v) = \lambda w$ and $\lambda w \in \text{range } T$.

\n\section*{Recommended Reading}\n\begin{itemize}\n  \item \textbf{Prüfungstraining Lineare Algebra : Band I} by Thomas C. T. Michaels (Match: 0.73)\n  \item \textbf{Lineare Algebra 1} by Menny-Akka (Match: 0.73)\n  \item \textbf{Prüfungstraining Lineare Algebra : Band II} by Thomas C. T. Michaels, Marcel Liechti (Match: 0.71)\n\end{itemize}\n
\end{document}