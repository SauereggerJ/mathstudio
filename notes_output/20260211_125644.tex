\documentclass{article}
\usepackage{amsmath, amssymb}
\begin{document}

\section*{Open Sets in a Metric Space $(M, d)$}

\noindent\textbf{1) Definition:}
Given $\varepsilon > 0$, the $\varepsilon$-neighborhood (nbhd) around $x \in M$ is the set
$$
B_{\varepsilon}(x) = \{y \in M : d(x, y) < \varepsilon\}
$$

\noindent\textbf{2) Definition:}
For any $A \subseteq M$, we say $A$ is open if for any $x \in A$ there exists an $\varepsilon > 0$ so that the $\varepsilon$-neighborhood of $x$ is a subset of $A$.
$$
\forall x \in A: \exists \varepsilon > 0: B_{\varepsilon}(x) \subseteq A
$$

\noindent\textbf{3) Definition:}
A neighborhood of $A \subseteq M$, $x \in M$, is any open set containing $x$.

\noindent\textbf{Lemma:}
For any $x \in M$ and $\varepsilon > 0$, $B_{\varepsilon}(x)$ is open.

\noindent\textbf{Proof:}
Given any $x \in M$ and $\varepsilon > 0$, choose $z \in B_{\varepsilon}(x)$.
Define $\delta_z = \varepsilon - d(x, z)$. Consider any $y \in B_{\delta_z}(z)$,
notice
$$d(x, y) \le d(x, z) + d(z, y) < d(x, z) + \delta_z = d(x, z) + (\varepsilon - d(x, z)) = \varepsilon$$
Thus $y \in B_{\varepsilon}(x)$ and therefore $B_{\delta_z}(z) \subseteq B_{\varepsilon}(x)$. $\square$

\noindent\textbf{Lemma:}
In a metric space $(M, d)$:
\begin{enumerate}
    \item $M$ and $\emptyset$ are open.
    \item If $A_{\lambda}$ is open for each index $\lambda \in \Lambda$, then $\bigcup_{\lambda \in \Lambda} A_{\lambda}$ is open.
    \item If $A_1, \dots, A_N$ are open, then $\bigcap_{k=1}^{N} A_k$ is open.
\end{enumerate}

\noindent\textbf{Proof:}
\begin{enumerate}
    \setcounter{enumi}{1}
    \item Suppose $x \in \bigcup_{\lambda \in \Lambda} A_{\lambda}$. Then $\exists \lambda_0 \in \Lambda$ with $x \in A_{\lambda_0}$. Since $A_{\lambda_0}$ is open, $\exists \varepsilon > 0$ such that $B_{\varepsilon}(x) \subseteq A_{\lambda_0} \subseteq \bigcup_{\lambda \in \Lambda} A_{\lambda}$.
    \item Suppose $x \in \bigcap_{k=1}^{N} A_k$. For each $k \in \{1, \dots, N\}$, $x \in A_k$, and $A_k$ is open: $\exists \varepsilon_k > 0: B_{\varepsilon_k}(x) \subseteq A_k$.
    Let $\varepsilon = \min \{\varepsilon_1, \dots, \varepsilon_N\}$. Then $B_{\varepsilon}(x) \subseteq A_k$ for all $k \in \{1, \dots, N\}$.
    Thus $B_{\varepsilon}(x) \subseteq \bigcap_{k=1}^{N} A_k$. $\square$
\end{enumerate}
\n\section*{Recommended Reading}\n\begin{itemize}\n  \item \textbf{Writing Proofs in Analysis} by Jonathan M. Kane (Match: 0.70)\n  \item \textbf{Basic Analysis I} by James K. Peterson (Match: 0.68)\n  \item \textbf{Real Analysis} by Patrick Fitzpatrick (Match: 0.68)\n\end{itemize}\n
\end{document}