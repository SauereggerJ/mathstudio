\documentclass{article}
\usepackage{amsmath}
\begin{document}

Sei $f(x)$ eine auf einem gegebenen Intervall $I$ definierte reelle Funktion.
Falls $f(x)$ konvex ist gilt $f'(x)$ ist wachsend, entsprechendes f\"ur konkav und fallend.

\textbf{Beweis:}

Sei $a, b \in I$ mit $a < b$, so gilt es eine Folge $x_n \to a$ und eine Folge $y_n \to b$
in $I$ so dass $a \le x_n < x_0 < y_0 < y_n \le b$ dann gilt

$$\frac{f(x_n)-f(a)}{x_n-a} < \frac{f(x_0)-f(a)}{x_0-a} < \frac{f(y_0)-f(x_0)}{y_0-x_0} < \frac{f(y_n)-f(y_0)}{y_n-y_0} < \frac{f(y_n)-f(b)}{y_n-b}$$

Durch $n \to \infty$ erh\"alt man

$$f'(a) < \frac{f(x_0)-f(a)}{x_0-a} < \frac{f(y_0)-f(x_0)}{y_0-x_0} \le f'(b)$$

also ist $f$ in $I$ streng monoton wachsend.

Die Umkehrung dieser Aussage ist auch richtig.

\textbf{Beweis:}
Es seien $a, x, b \in I$ mit $a < x < b$. Mit dem Mittelwertsatz gilt es dann
ein $\xi \in (a, x)$ und ein $\eta \in (x, b)$ mit:

$$\frac{f(x)-f(a)}{x-a} = f'(\xi) \quad \text{und} \quad \frac{f(b)-f(x)}{b-x} = f'(\eta).$$

also mit der strengen Monotonie von $f'$ gilt:
$$\frac{f(x)-f(a)}{x-a} < \frac{f(b)-f(x)}{b-x} \quad \text{und}$$ 

somit ist $f$ konvex.

$0,3$ ist schon sehr fein, also gut zum klein schreiben.

\n\section*{Recommended Reading}\n\begin{itemize}\n  \item \textbf{Tutorium Analysis 1 und Lineare Algebra 1} by Florian Modler, Martin Kreh (Match: 0.67)\n  \item \textbf{Grundkurs Analysis 2} by Klaus Fritzsche (Match: 0.67)\n  \item \textbf{Analysis I} by H. Amann (Match: 0.67)\n\end{itemize}\n
\end{document}