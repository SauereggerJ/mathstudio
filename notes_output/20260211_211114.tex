\documentclass{article}\usepackage{amsmath}\begin{document}

\section*{Problems}

1) Prove that $-(-v) = v$ for every $v \in V$.
With $1.32$: $(-1)v = -v$
\[
-(-v) = (-1)(-v) = (-1)((-1)v) = (-1)(-1)v = 1v = v
\]

2) Suppose $\alpha \in F$, $v \in V$, and $\alpha v = 0$. Prove that $\alpha = 0$ or $v = 0$.
If $\alpha \neq 0$, then there exists $\alpha^{-1}$ with $\alpha \alpha^{-1} = 1$,
\[
\alpha^{-1}(\alpha v) = \alpha^{-1} \cdot 0
\\
(\alpha^{-1}\alpha) v = 0
\\
1v = 0
\\
v = 0
\]
If $v \neq 0$, then $\alpha$ has to be zero, otherwise it will contradict the previous part.

3) Suppose $v, w \in V$. Explain why there exists a unique $x \in V$ such that $v + 3x = w$.
1) For $(w-v) \cdot \frac{1}{3} = x$ the statement is true.
2) If $x'$ is an other element of $V$ with $x' \neq (w-v) \cdot \frac{1}{3}$ and
\[
v + 3x' = w : \quad v + 3x' = v + 3x
\\
3x' = 3x
\\
x' = x
\]
So it's because of $a=b \iff a+x = b+x$, and $a=b \iff x \cdot a = x \cdot b$ for $x \in F$ and $x \neq 0$.

4) The empty set is no vector space because of:
\[
\exists x \in V : v+x=x \text{ for all } v \in V.
\]

\n\section*{Recommended Reading}\n\begin{itemize}\n  \item \textbf{Linear algebra problem book} by Paul R. Halmos (Match: 0.69)\n  \item \textbf{Prüfungstraining Lineare Algebra : Band I} by Thomas C. T. Michaels (Match: 0.68)\n  \item \textbf{Lineare Algebra 1} by Menny-Akka (Match: 0.68)\n\end{itemize}\n
\end{document}