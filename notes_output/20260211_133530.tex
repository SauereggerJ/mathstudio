\documentclass{article}
\usepackage{amsmath}
\usepackage{amssymb}
\begin{document}

\noindent 22.10. Algebra.\

\noindent \textbf{Definition:} A number $\lambda \in \mathbb{F}$ is called a zero (or root) of a polynomial $p \in \mathbb{P}(\mathbb{F})$ if
$$
 p(\lambda) = 0.
$$

\noindent \textbf{Lemma:} Suppose $m$ is a positive integer and $p \in \mathbb{P}(\mathbb{F})$ is a polynomial of degree $m$. Suppose $\lambda \in \mathbb{F}$. Then $p(\lambda) = 0$ if and only if there exists a polynomial $q \in \mathbb{P}(\mathbb{F})$ of degree $m-1$ such that
$$ p(z) = (z - \lambda) \cdot q(z). $$

\noindent \textbf{Proof:} $\Rightarrow)$ Suppose $p(\lambda) = 0$. And $p(z) = a_0 + a_1 z + \dots + a_n z^n$ for all $z \in \mathbb{F}$.
Then $p(z) = p(z) - p(\lambda) = a_0 + a_1 z + \dots + a_n z^n - (a_0 + a_1 \lambda + \dots + a_n \lambda^n)$.
For each $k \in \{1, \dots, m\}$, the equation $z^k - \lambda^k = (z - \lambda) \cdot \sum_{j=0}^{k-1} z^j \lambda^{k-1-j}$ holds. Thus $p(z) = (z - \lambda) \cdot q(z)$ for some polynomial $q(z)$ of degree $m-1$.

\$\Leftarrow$ Suppose such a $q(z)$ exists. Then $p(\lambda) = (\lambda - \lambda) \cdot q(\lambda) = 0 \cdot q(\lambda) = 0$. $\square$

\noindent \textbf{Lemma:} Suppose $m$ is a positive integer and $p \in \mathbb{P}(\mathbb{F})$ is a polynomial of degree $m$. Then $p$ has at most $m$ zeros in $\mathbb{F}$.

\noindent \textbf{Proof:} By induction: $m=1$: because $a_1 \neq 0$, $z = -a_0/a_1$ is the only zero.

$m-1 \Rightarrow m$: If $p$ has no zeros in $\mathbb{F}$, nothing is to show. If $\lambda$ is a zero of $p$, then $q \in \mathbb{P}(\mathbb{F})$ exists with degree $m-1$ such that $p(x) = (x - \lambda) \cdot q(x)$. $q(x)$ has at most $m-1$ zeros, and therefore $p(x)$ at most $m$ zeros. $\square$

\noindent \textbf{Remark:} This implies that the coefficients of a polynomial are uniquely determined, because if a polynomial had two different sets of coefficients, then subtracting the two representations would give a polynomial with non-zero degree but infinitely many zeros.

$(\text{``} \text{Herausheben von NS''} \text{ \& max. } m \text{ zeros})$
$$ \rightarrow \text{Grad} - 1$$

\n\section*{Recommended Reading}\n\begin{itemize}\n  \item \textbf{Abstract Algebra} by Paul B. Garrett (Match: 0.69)\n  \item \textbf{Abstract Algebra} by Marco Hien (Match: 0.69)\n  \item \textbf{Lineare Algebra 1} by Menny-Akka (Match: 0.69)\n\end{itemize}\n
\end{document}