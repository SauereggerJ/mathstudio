\documentclass{article}
\usepackage[utf8]{inputenc}
\usepackage{amsmath, amssymb, amsfonts}
\begin{document}
\title{Interior and Closed Sets in Metric Spaces}
\maketitle
\section*{The Interior of a Set in a Metric Space $(M,d)$}

\begin{definition} 
Let $A \subseteq M$. A point $x \in A$ is an interior point of $A$ if there exists an open set $U$ such that $x \in U \subseteq A$.
\end{definition}

The interior of $A$, $\text{int}(A)$, is the set of all interior points of $A$.

\textbf{Example:}
$$
\text{int}(\mathbb{R}^n) = \mathbb{R}^n, \quad \text{int}((0,1)) = (0,1), \quad \text{int}(\Gamma \cup \{1\}) = (0,1),
$$
$$
\text{int}(\mathbb{Z}) = \emptyset. \text{ If } A \subseteq \mathbb{R} \text{ is any set such that } d \text{ is the discrete metric, } \text{int}(A) = A.
$$

\begin{lemma} 
The interior of $A \subseteq M$ is the union of all open subsets of $A$.
\end{lemma}

\textbf{Proof:} Let $\mathcal{B}$ call the union of all open subsets of $A$.
If $x \in \text{int}(A) \implies \exists$ open $U$ such that $x \in U \subseteq A$.
If $x \in \mathcal{B} \implies x$ lies in one element of $\mathcal{B} \implies x \in \text{int}(A)$.

\begin{corollary} 
$\text{int}(A)$ is open.
\end{corollary}

\begin{corollary} 
$A$ is open $\iff \text{int}(A) = A$.
\end{corollary}

\textbf{Proof:}
$(\Rightarrow)$: Suppose $A$ is open. Then for all $x \in A$, $x \in A \subseteq A$.
So $x \in \text{int}(A)$. Thus $A \subseteq \text{int}(A)$. Since $\text{int}(A) \subseteq A$, we have $\text{int}(A) \subseteq A$.
$(\Leftarrow)$: If $A = \text{int}(A)$ and $\text{int}(A)$ is open $\implies A$ is open.

\section*{Closed Sets in a Metric Space $(M,d)$}

\begin{definition} 
Let $(M,d)$ be a metric space. A set $B \subseteq M$ is closed if its complement $M \setminus B$ is open.
\end{definition}

\begin{lemma} 
In a metric space $(M,d)$:
\begin{enumerate}
    \item $M$ and $\emptyset$ are closed.
    \item If $A_{\lambda}$ is closed for each index $\lambda \in \Lambda$, then $\bigcap_{\lambda \in \Lambda} A_{\lambda}$ is closed.
    \item If $A_1, \dots, A_N$ are closed, then $\bigcup_{k=1}^N A_k$ is closed.
\end{enumerate}
\end{lemma}

\textbf{Proof:}
\begin{enumerate}
    \item $M^c = \emptyset$; $\emptyset^c = M$.
    \item $A_{\lambda}$ closed $\implies A_{\lambda}^c$ open. $\bigcap_{\lambda \in \Lambda} A_{\lambda} = \left(\bigcup_{\lambda \in \Lambda} A_{\lambda}^c\right)^c$ closed;
    \item $A_1, \dots, A_N$ closed $\implies \bigcap_{i=1}^N A_i^c$ open; $\bigcup_{i=1}^N A_i = \left(\bigcap_{i=1}^N A_i^c\right)^c$ closed.
\end{enumerate}
\n\section*{Recommended Reading}\n\begin{itemize}\n  \item \textbf{Writing Proofs in Analysis} by Jonathan M. Kane (Match: 0.69)\n  \item \textbf{Introduction to Real Analysis} by Christopher Heil (Match: 0.68)\n  \item \textbf{Real Analysis} by Patrick Fitzpatrick (Match: 0.67)\n\end{itemize}\n
\end{document}