\documentclass{article}
\usepackage{amsmath, amssymb}
\begin{document}

\section*{Series 1}
$E$ stands for a Banach space $(E, \|\cdot\|)$. $\mathbb{K}$ is $\mathbb{R}$ or $\mathbb{C}$.

\noindent\textbf{Definition:}
Let $(x_k)$ be a sequence in $E$. Then $S_n := \sum_{k=0}^{n} x_k$ for $n \in \mathbb{N}$ defines a new sequence $(S_n)$ in $E$, called the series in $E$.
The element $S_n$ is called the $n$-th partial sum, and $x_n$ is called the $n$-th summand of the series $\sum x_k$.

The series $\sum x_k$ converges if $(S_n)$ converges. The limit of $(S_n)$ is called the value of the series $\sum x_k$ and is written $\sum_{k=0}^{\infty} x_k$. The series diverges if the sequence $(S_n)$ diverges in $E$.

\noindent\textbf{Proposition:}
If the series $\sum x_k$ converges, then $(x_k)$ is a null sequence.
\noindent\textbf{Proof:}
Because $(S_n)$ converges, $(S_n)$ is a Cauchy sequence: $\forall \varepsilon > 0, \exists N \in \mathbb{N}$ such that $|S_n - S_m| < \varepsilon$ for all $m, n \geq N$. In particular, $|S_{n+1} - S_n| = |\sum_{k=0}^{n+1} x_k - \sum_{k=0}^{n} x_k| = |x_{n+1}| < \varepsilon$ for large enough $n$.

\medskip
\noindent\textbf{Convergence Tests:}
\noindent\textbf{Theorem (Cauchy criterion):}
For a series $\sum x_k$ in $E$ the following are equivalent:
\begin{enumerate}
    \item $\sum x_k$ converges.
    \item For each $\varepsilon > 0$, there is some $N \in \mathbb{N}$ such that $|\sum_{k=n+1}^{m} x_k| < \varepsilon$ for $m > n \geq N$.
\end{enumerate}
\noindent\textbf{Proof:}
$S_m - S_n = \sum_{k=n+1}^{m} x_k$. Thus $(S_n)$ is a Cauchy sequence in $E$ if and only if (ii) is true, and in a Banach space a Cauchy sequence converges.

\noindent\textbf{Theorem:}
Let $\sum x_k$ be a series in $\mathbb{R}$ such that $x_k > 0$ for all $k \in \mathbb{N}$. Then $\sum x_k$ converges if and only if $(S_n)$ is bounded. In this case, the series has the value $\sup_{n \in \mathbb{N}} S_n$.
\noindent\textbf{Proof:}
Since the summands are nonnegative, the sequence $(S_n)$ is increasing, by the Bolzano-Weierstrass theorem $(S_n)$ only converges if $(S_n)$ is bounded.

\noindent\textbf{Definition:}
The series $\sum x_k$ converges absolutely or is absolutely convergent if $\sum |x_k|$ converges in $\mathbb{R}$ (that is $\sum_{k=0}^{\infty} |x_k| < \infty$).

\n\section*{Recommended Reading}\n\begin{itemize}\n  \item \textbf{Analysis I} by H. Amann (Match: 0.71)\n  \item \textbf{Writing Proofs in Analysis} by Jonathan M. Kane (Match: 0.70)\n  \item \textbf{Analysis in Banach Spaces : Volume I} by Tuomas Hytönen, Jan van Neerven, Mark Veraar, Lutz Weis (Match: 0.70)\n\end{itemize}\n
\end{document}