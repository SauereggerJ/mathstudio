\documentclass{article}
\usepackage{amsmath, amssymb}
\begin{document}

\title{Landau Symbole}
\author{}
\date{}
\maketitle

Seien $X$ und $E$ normierte Vektorräume, $D$ eine nichtleere Teilmenge von $X$ und $f: D \to E$ eine norm. Vektorraum.

\textbf{Definition:} Sei $\alpha \in \overline{D}$, ist $\alpha \ge 0$ so sagen wir ``$f$ verschwindet in $\alpha$ von höherer Ordnung als $\alpha$-ten'' und schreiben:
$$ f(x) = o\left( \left\lVert x - \alpha \right\rVert^{\alpha} \right) \quad (x \to \alpha) $$


$$ \lim_{x \to \alpha} \frac{f(x)}{\left\lVert x - \alpha \right\rVert^{\alpha}} = 0 $$


\textbf{Lemma:} $\left( \forall \varepsilon > 0: \exists U \subseteq D \text{ Umgebung von } \alpha: \left\lVert f(x) \right\rVert \le \varepsilon \left\lVert x - \alpha \right\rVert^{\alpha}, x \in U \right) \iff f(x) = o\left( \left\lVert x - \alpha \right\rVert^{\alpha} \right) \quad (x \to \alpha)$

\textbf{Beweis:} $\Rightarrow$ Sei $U_{\alpha} = U \setminus \{ \alpha \}$ die punktierte Umgebung von $\alpha$ für eine belieb. Umgebung $U$ von $\alpha$. 
Es gilt dann $\forall \varepsilon > 0: \exists U_{\alpha} \subset D: \left\lVert \frac{f(x)}{\left\lVert x - \alpha \right\rVert^{\alpha}} \right\rVert < \varepsilon, x \in U_{\alpha}$, das
bedeutet aber $\lim_{x \to \alpha} \frac{\left\lVert f(x) \right\rVert}{\left\lVert x - \alpha \right\rVert^{\alpha}} = 0$ und hieraus folgt $\lim_{x \to \alpha} \frac{f(x)}{\left\lVert x - \alpha \right\rVert^{\alpha}} = 0$.


$\Leftarrow$ Umgekehrt folgt aus $\lim_{x \to \alpha} \frac{f(x)}{\left\lVert x - \alpha \right\rVert^{\alpha}} = 0$:
$$ g(x) := \frac{f(x)}{\left\lVert x - \alpha \right\rVert^{\alpha}}, \quad \forall V_0 \subset E: \exists U_{\alpha} \subset X: g(U_{\alpha} \cap D) \subset V_0, \text{ das ist} $$

Äquivalent zu
$$ \left\lVert g(x) \right\rVert \le \varepsilon \quad \forall x \in U_{\alpha} \quad \text{ und somit }$$
$$ \left\lVert f(x) \right\rVert \le \varepsilon \left\lVert x - \alpha \right\rVert^{\alpha}, \quad x \in U_{\alpha}. $$

\textbf{Lemma:} Es sei $X = \mathbb{R}^n$. Dann ist $f: D \to E$ genau dann in $\alpha \in D$ differenzierbar, wenn es ein (eindeutig bestimmtes) $m_{\alpha} \in E$ gibt mit:
$$ f(x) - f(\alpha) = m_{\alpha}(x - \alpha) + o(\left\lVert x - \alpha \right\rVert) \quad (x \to \alpha) $$

\textbf{Beweis:} $\Leftarrow$ Laut Definition gilt dann $\lim_{x \to \alpha} \frac{f(x) - f(\alpha) - m_{\alpha}(x - \alpha)}{\left\lVert x - \alpha \right\rVert} = 0$

also 
$$ \lim_{x \to \alpha} \frac{f(x) - f(\alpha)}{\left\lVert x - \alpha \right\rVert} - m_{\alpha} \lim_{x \to \alpha} \frac{(x - \alpha)}{\left\lVert x - \alpha \right\rVert} = 0 $$

$$ \lim_{x \to \alpha} \frac{f(x) - f(\alpha)}{\left\lVert x - \alpha \right\rVert} = m_{\alpha} \lim_{x \to \alpha} \frac{(x - \alpha)}{\left\lVert x - \alpha \right\rVert} \Rightarrow \lim_{x \to \alpha} \frac{f(x) - f(\alpha)}{\left\lVert x - \alpha \right\rVert} = m_{\alpha} $$

$\Rightarrow$: differenzierbar ist äquivalent zu: Also es gilt $m_{\alpha}$ und $r(x): X \to E$
such, dass $r$ stetig und $r(\alpha) = 0$ und: $f(x) = f(\alpha) + m_{\alpha}(x - \alpha) + r(x)(x - \alpha)$ für alle $x \in \overline{D}$. Also 
$$ \lim_{x \to \alpha} \frac{f(x) - f(\alpha) - m_{\alpha}(x - \alpha)}{\left\lVert x - \alpha \right\rVert} = 0 \dots $$

\n\section*{Recommended Reading}\n\begin{itemize}\n  \item \textbf{Lineare Funktionalanalysis Eine Anwendungsorientierte Einfhrung} by Hans Wilhelm Alt (Match: 0.73)\n  \item \textbf{Grundkurs Analysis 2} by Klaus Fritzsche (Match: 0.73)\n  \item \textbf{Angewandte Funktionalanalysis} by Manfred Dobrowolski (Match: 0.71)\n\end{itemize}\n
\end{document}