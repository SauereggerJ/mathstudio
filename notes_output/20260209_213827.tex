\documentclass{article}
\usepackage{amsmath}

\begin{document}

\textbf{Definition:} A linear map is a function $T: V \to W$, with the following properties:
\begin{enumerate}
    \item For all $u, v \in V$: $T(u+v) = T(u) + T(v)$
    \item For all $v \in V$ and $\lambda \in K$: $T(\lambda v) = \lambda T(v)$
\end{enumerate}

\textbf{Linear Map Lemma:}
If $v_1, ..., v_n$ is a basis of $V$, and $w_1, ..., w_n \in W$, then there exists a unique linear map $f$ from $V$ to $W$ with $f(v_i) = w_i$, for all $i \in \{1, ..., n\}$.

\textbf{Proof:}
Every $v \in V$ can be written as $v = v_1 c_1 + ... + v_n c_n$. We define $f: V \to W$ such that $f(v) = c_1 w_1 + ... + c_n w_n$.
If we take $c_i = 1$ and $c_j = 0$ for $j \neq i$, we have $f(v_i) = w_i$, so $f$ fulfills the hypothesis.
We now show that $f$ is a linear function:
With $v = v_1 c_1 + ... + v_n c_n$ and $u = v_1 \overline{c_1} + ... + v_n \overline{c_n}$, we have
\begin{align*}
f(v+u) &= w_1(c_1 + \overline{c_1}) + ... + w_n(c_n + \overline{c_n}) = w_1 c_1 + ... + w_n c_n + w_1 \overline{c_1} + ... + w_n \overline{c_n} \\
&= f(v) + f(u)
\end{align*}
And with $\lambda \in K$,
\begin{align*}
f(\lambda v) &= w_1 \lambda c_1 + ... + w_n \lambda c_n = \lambda(w_1 c_1 + ... + w_n c_n) = \lambda f(v).
\end{align*}
The uniqueness of $f$: Let $\tilde{f}$ be a linear function with the $f(v_i) = w_i$ property. Then because $v_1, ..., v_n$ is a basis, we have
\begin{align*}
f(v) &= f(c_1 v_1 + ... + c_n v_n) = f(v_1) c_1 + ... + f(v_n) c_n = w_1 c_1 + ... + w_n c_n.
\end{align*}
This means $f = \tilde{f}$ for all $v \in V$.

\textbf{Definition:} If $V$ and $W$ are vector spaces, we define $\mathcal{L}(V, W)$ as the set of all linear functions from $V$ to $W$.
Suppose $S, T \in \mathcal{L}(V, W)$ and $\lambda \in F$. We define the sum $S+T$ and the product $\lambda T$ by:
\begin{align*}
(S+T)(v) &= S(v) + T(v) \text{ and } \\
(\lambda T)(v) &= \lambda T(v).
\end{align*}

Then $\mathcal{L}(V, W)$ is a vector space.

\n\section*{Recommended Reading}\n\begin{itemize}\n  \item \textbf{Lineare Algebra 1} by Menny-Akka (Match: 0.71)\n  \item \textbf{Prüfungstraining Lineare Algebra : Band I} by Thomas C. T. Michaels (Match: 0.70)\n  \item \textbf{Lineare Algebra} by Dirk Werner (Match: 0.70)\n\end{itemize}\n
\end{document}