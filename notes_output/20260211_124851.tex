\documentclass{article}
\usepackage{amsmath, amssymb}
\title{Differentiability 1}
\author{}
\date{}
\begin{document}
\maketitle

\section*{Conventions}
$X \subseteq \mathbb{K}$ is a set, $a \in X$ is a limit point of $X$ and $E = (E, \|\cdot\|_{E})$ is a normed vector space over $\mathbb{K}$.

\subsection*{The Derivative}
\textbf{Def:} A function $f: X \to E$ is called differentiable at $a$ if the limit
$$f'(a) := \lim_{x \to a} \frac{f(x) - f(a)}{x - a}$$
exists exists in $E$. When this occurs, $f'(a) \in E$ is called the derivative of $f$ at $a$. Besides the symbols $f'(a)$ many other notations for the derivative are used:
$$\dot{f}(a), \quad \partial f(a), \quad Df(a), \quad \frac{df}{dx}(a)$$

\textbf{Theorem:} For $f: X \to E$ the following are equivalent:
\begin{enumerate}
    \item $f$ is differentiable at $a$
    \item There is some $m_a \in E$ such that
    $$\lim_{x \to a} \frac{f(x) - f(a) - m_a(x - a)}{x - a} = 0$$
    \item There are $m_a \in E$ and a function $r: X \to E$ which is continuous at $a$ such that $r(a) = 0$ and
    $$f(x) = f(a) + m_a(x - a) + r(x)(x - a), \quad x \in X$$
\end{enumerate}
in cases (ii) and (iii) $\implies m_a = f'(a)$.

\textbf{Proof:}
\begin{enumerate}
    \item[i) $\Rightarrow$ ii)] $m_a := f'(a)$.
    $$\lim_{x \to a} \frac{f(x) - f(a)}{x - a} - \frac{m_a(x - a)}{x - a} = m_a - m_a = 0$$
    \item[ii) $\Rightarrow$ iii)] Define $r(x):=$
    $$\begin{cases} 0 & , x = a \\ \frac{f(x) - f(a) - m_a(x - a)}{x - a} & , x \ne a \end{cases}$$
because of (ii) $\lim_{x \to a} r(x) = 0 = r(a)$ and so $r$ is continuous at $a$.

We need to show $f(x) = f(a) + m_a(x - a) + r(x)(x - a)$.

For $x=a$:
$$f(a) + m_a(a-a) + r(a)(a-a) = f(a) + 0 + 0 = f(a)$$

For $x \ne a$:
\begin{align*}
& f(a) + m_a(x-a) + r(x)(x-a) \\ &= f(a) + m_a(x-a) + \left( \frac{f(x) - f(a) - m_a(x - a)}{x - a} \right) (x - a) \\ &= f(a) + m_a(x-a) + f(x) - f(a) - m_a(x-a) \\ &= f(x) \quad \text{for } x \ne a.
\end{align*}

    \item[iii) $\Rightarrow$ i)] From (iii):
    $$\frac{f(x) - f(a)}{x - a} = m_a + r(x)$$
    Taking the limit as $x \to a$:
    $$\lim_{x \to a} \frac{f(x) - f(a)}{x - a} = \lim_{x \to a} m_a + \lim_{x \to a} r(x) = m_a$$ because $r(x)$ is continuous and $r(a) = 0$.
\end{enumerate}

\n\section*{Recommended Reading}\n\begin{itemize}\n  \item \textbf{Analysis I} by H. Amann (Match: 0.71)\n  \item \textbf{Fréchet differentiability of Lipschitz functions and porous sets in Banach spaces} by Joram Lindenstrauss (Match: 0.71)\n  \item \textbf{Calculus On Normed Vector Spaces} by Rodney Coleman (Match: 0.70)\n\end{itemize}\n
\end{document}