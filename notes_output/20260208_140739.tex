\documentclass{article}
\usepackage{amsmath}

\begin{document}

\noindent Definition: For $T \in \mathcal{L}(V, W)$, the null space of $T$ is the subset of $V$ with $T(v) = 0$. \\ 
null $T = \{v \in V \mid T(v) = 0 \}$

\noindent Definition: A function $T: V \to W$ is called injective if $T(v) = T(w)$ implies $w=v$, or equivalently, if $v \neq w$ then $T(v) \neq T(w)$.

\noindent Definition: For $T \in \mathcal{L}(V, W)$ the range of $T$ is the subset of $W$ consisting of vectors that are equal to $T(v)$ for some $v \in V$. \\
range $T = \{w \in W \mid \exists v \in V 	ext{ s.t. } T(v) = w \}$

\noindent Definition: A function $T: V \to W$ is called surjective if $W$ = range $T$.

\noindent Theorem: Suppose $T \in \mathcal{L}(V, W)$, then null $T$ is a subspace of $V$.

\noindent Proof: Because of $T(0) = 0$, $0$ is an element of null $T$. For $v, w \in $ null $T$ we have $T(v+w) = T(v) + T(w) = 0 + 0 = 0$, so $v+w \in$ null $T$, and with $\lambda \in \mathbb{K}$ we get $T(\lambda v) = \lambda T(v) = \lambda \cdot 0 = 0$, so $ \lambda v$ is also an element of null $T$, and null $T$ therefore is a subspace of $V$.

\noindent Theorem: For $T \in \mathcal{L}(V, W)$, null $T = \{0\}$ if and only if $T$ is injective.

\noindent Proof: Suppose $T$ is injective and $v \in V$ is an element of null $T$, then $T(v) = 0 = T(0)$ and because of injectivity, $v = 0$. Therefore null $T = \{0\}$.

\noindent Suppose null $T = \{0\}$ and $T(v) = T(w)$ for two elements $v, w$ of $V$, then $T(v) - T(w) = T(v-w) = 0$, so $v-w \in$ null $T$, but so $v-w$ must be $0$, and this yields $v = w$, therefore $T$ is injective.

\noindent Lemma (because we used it once already): $0 \in$ null $T$.

\noindent Proof: $T(0) = T(v-v) = T(v) - T(v) = 0$.

\noindent Theorem: Suppose $T \in \mathcal{L}(V, W)$, then range $T$ is a subspace of $W$.

\noindent Proof: \\ 1.) $0 \in$ range $W$ because $T(0) = 0$. If $w_1, w_2 \in$ range $T$, then there exist $v_1, v_2$ with $T(v_1) = w_1$ and $T(v_2) = w_2$ and so $T(v_1 + v_2) = T(v_1) + T(v_2) = w_1 + w_2$, therefore $w_1 + w_2 \in$ range $W$. If $w \in$ range $T$ and $\lambda \in \mathbb{K}$, then there exist a $v \in V$ with $T(v) = w$, and so $T(\lambda v) = \lambda T(v) = \lambda w$ and $ \lambda w \in$ range $T$.

\n\section*{Recommended Reading}\n\begin{itemize}\n  \item \textbf{Prüfungstraining Lineare Algebra : Band I} by Thomas C. T. Michaels (Match: 0.72)\n  \item \textbf{Lineare Algebra 1} by Menny-Akka (Match: 0.72)\n  \item \textbf{Linear Algebra} by Meckes & Meckes (Match: 0.71)\n\end{itemize}\n
\end{document}