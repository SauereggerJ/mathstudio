\documentclass{article}
\usepackage{amsmath}
\usepackage{amsfonts}
\usepackage{amssymb}

\begin{document}

64 Chapter 3 Linear Maps

The next result shows that no linear map from a finite-dimensional vector
space to a “bigger” vector space can be surjective, where “bigger” is measured by
dimension.

3.24 linear map to a higher-dimensional space is not surjective
Suppose $V$ and $W$ are finite-dimensional vector spaces such that
$\text{dim } V < \text{dim } W$. Then no linear map from $V$ to $W$ is surjective.

Proof Let $T \in \mathcal{L}(V, W)$. Then
$\text{dim range } T = \text{dim } V - \text{dim null } T$
$\leq \text{dim } V$
$< \text{dim } W$,
where the equality above comes from the fundamental theorem of linear maps
(3.21). The inequality above states that $\text{dim range } T < \text{dim } W$. This means that
$\text{range } T$ cannot equal $W$. Thus $T$ is not surjective.

As we will soon see, 3.22 and 3.24 have important consequences in the theory
of linear equations. The idea is to express questions about systems of linear
equations in terms of linear maps. Let’s begin by rephrasing in terms of linear
maps the question of whether a homogeneous system of linear equations has a
nonzero solution.

Homogeneous, in this context, means
that the constant term on the right side
of each equation below is 0.

Fix positive integers $m$ and $n$, and let
$A_{j, k} \in \mathbb{F}$ for $j = 1 , …, m$ and $k = 1 , …, n$.
Consider the homogeneous system of lin-\near equations
$$\sum_{k = 1}^{n} A_{1, k} x_k = 0$$
$$\vdots$$
$$\sum_{k = 1}^{n} A_{m, k} x_k = 0.$$
Clearly $x_1 = \cdots = x_n = 0$ is a solution of the system of equations above; the
question here is whether any other solutions exist.

Define $T\colon \mathbb{F}^n \to \mathbb{F}^m$ by
3.25 $T(x_1, …, x_n) = (\sum_{k = 1}^{n} A_{1, k} x_k, …, \sum_{k = 1}^{n} A_{m, k} x_k)$.

The equation $T(x_1, …, x_n) = 0$ (the 0 here is the additive identity in $\mathbb{F}^m$, namely,
the list of length $m$ of all 0’s) is the same as the homogeneous system of linear
equations above.

Thus we want to know if $\text{null } T$ is strictly bigger than $\{0\}$, which is equivalent
to $T$ not being injective (by 3.15). The next result gives an important condition
for ensuring that $T$ is not injective.

\end{document}