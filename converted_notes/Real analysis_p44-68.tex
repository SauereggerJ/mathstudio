
% --- Page 44 ---
\documentclass{article}
\usepackage{amsmath, amssymb, amsthm}
\begin{document}

1.7. Some Linear Algebra 23

7. $\dim(\mathcal{P}_j)$ is called the algebraic multiplicity of $\lambda_j$. $\dim[\text{Ker}(N_j) \cap \text{Ran}(P_j)] = \dim\{x| Tx = \lambda x\}$ is the geometric multiplicity.

8. The $P_j$ are called spectral projections (although some reserve that for the case of self-adjoint matrices). They are also called eigenprojections and the $N$'s eigennilpotents.

We need (in Section 7.5) a result about the form of rank one projections in $\mathbb{C}^n$ and transposes in the Jordan normal form. For any matrix $A$, the transpose, $A^t$, is the matrix
$$(A^t)_{ij} = A_{ji} \quad (1.7.28)$$

\begin{proposition}
Let $A$ be an $n\times n$ complex matrix and
$$A = \sum_{k=1}^{m} (\lambda_k P_k + N_k) \quad (1.7.29)$$
its Jordan normal form. Then $A^t$ has the Jordan normal form
$$A^t = \sum_{k=1}^{m} \lambda_k P_k^t + N_k^t \quad (1.7.30)$$
If $\dim(P_{k_0})=1$ for some $k_0$ (which implies that $N_{k_0} = 0$), then there are vectors, $v, w$, obeying
$$Av = \lambda_{k_0}v, \quad A^t w = \lambda_{k_0}w \quad (1.7.31)$$
$$(P_{k_0})_{ij} = v_i w_j \quad (1.7.32)$$
$$\sum_{j} w_j v_j = 1 \quad (1.7.33)$$
\end{proposition}
\begin{proof}
$P_k P_\ell = \delta_{k\ell} P_k$ implies $(P^t_\ell)(P^t_k) = \delta_{k\ell} P^t_k$, $P_k N_k P_k = N_k$ implies $(P^t_k)(N^t_k)(P^t_k) = N^t_k$, and $(N_k)^m = 0$ implies $(N^t_k)^m = 0$. Thus, (1.7.30) is the Jordan normal form.

If $\dim(P_{k_0}) = 1$, (1.7.32) holds, where $v \in \text{Ran}(P)$ (so $Av = \lambda_{k_0}v$), and by looking at the adjoint, $w \in \text{Ran}(P^t)$ (so $A^t w = \lambda_{k_0}w$). $P^2 = P$ implies (1.7.33). $\square$
\end{proof}

\begin{definition}
A real inner product space is a real vector space, $X$, and a map $\langle\cdot ,\cdot\rangle : X\times X\to\mathbb{R}$ so that

(i) $\langle x,y\rangle =\langle y,x\rangle \quad (1.7.34)$
(ii) $\langle x,\cdot\rangle$ is linear in $\cdot$ for all $x\in X$.
(iii) $\langle x,x\rangle\geq 0$ and $\langle x,x\rangle = 0 \iff x = 0$.
\end{definition}

\end{document}
% --- Page 45 ---
% LaTeX generation failed because strict JSON was not returned.
% --- Page 46 ---
% LaTeX generation failed because strict JSON was not returned.
% --- Page 47 ---
% LaTeX generation failed because strict JSON was not returned.
% --- Page 65 ---
% LaTeX generation failed because strict JSON was not returned.
% --- Page 66 ---
\documentclass{article}
\usepackage{amsmath}
\usepackage{amssymb}
\begin{document}

\noindent in $(x^+,c)$, contradicting that $x^+$ is the sup. Thus, $B^- = \emptyset$. Similarly, $B^+ =\{x\in B\mid x>c \}$ is empty, so $B=\emptyset$. \hfill $\Box$

Here are some of the most significant properties associated to connectedness:
\begin{theorem}{2.1.13.}
(a) If $A\subset X$ is connected, so is $\overline{A}$.
(b) If $f: X\to Y$ is continuous and onto and $X$ is connected, so is $Y$.
(c) Any arcwise connected space is connected.
(d) If $\{A_\alpha\}_{\alpha\in I}$ is a collection of connected subsets of $X$, a topological space, so that for all $\alpha,\beta\in I$, $A_\alpha\cap A_\beta\neq\emptyset$, then $\cup_{\alpha\in I}A_\alpha$ is connected.
(e) If $x\in X$, a topological space, there is a unique connected $A\subset X$ with $x\in A$ that is maximal among all connected subsets of $X$ containing $x$. This $A$ is closed.
\end{theorem}

\noindent \textbf{Remarks.} 1. We’ll see below (Example 2.1.15) that connected spaces need not be arcwise connected.
2. The maximal connected $A \ni x$ is called a \textbf{connected component} of $X$. Lying in the same connected component is an equivalence relation.
3. Connected components need not be open in $X$ (see Example 2.1.14).

\noindent \textbf{Proof.} (a) Suppose $\overline{A} = B\cup C$ with $B\cap C =\emptyset$ and $B,C$ clopen in $\overline{A}$. Then $B\cap A$ and $C\cap A$ are clopen in $A$ so, since $A$ is connected, one is empty, that is, we can suppose $A\subset B$. Since $B$ is closed in $\overline{A}$, it is closed in $X$, and thus, $\overline{A}\subset B$ since $\overline{A}$ is the smallest closed set containing $A$. Thus, $\overline{A} = B$ and $C =\emptyset$.
(b) If $B,C$ are clopen in $Y$, $B\cap C =\emptyset$, and $B\neq\emptyset$, then $f^{-1}[B]\cap f^{-1}[C]= \emptyset$ and both sets are clopen in $X$. Since $X$ is connected, one can see $f^{-1}[C]$ must be empty. Since $f$ is onto $Y$, $C$ must be empty.
(c) Suppose $X$ is arcwise connected, $B,C$ are clopen in $X$ and disjoint and nonempty and $B\cup C = X$. Pick $x\in B, y \in C$, and $\gamma:[0,1] \to X$ with $\gamma(0) =x, \gamma(1) =y$. $\gamma^{-1}[B], \gamma^{-1}[C]$ are clopen and disjoint and cover $[0,1]$. But $0\in\gamma^{-1}[B], 1\in\gamma^{-1}[C]$, and $[0,1]$ is connected. This is a contradiction, so one of $B$ or $C$ is empty.
(d) Let $B,C$ be clopen in $Y =\cup_\alpha A_\alpha$ and disjoint with $B\cup C = Y$. Then for each $\alpha$, $B\cap A_\alpha, C\cap A_\alpha$ are clopen in $A_\alpha$ and disjoint, so since $A_\alpha$ is connected, for each $A_\alpha$, either $A_\alpha\subset B$ or $A_\alpha\subset C$. Suppose that for some $\alpha,\beta$, $A_\alpha\subset B, A_\beta\subset C$. Then, since $B\cap C =\emptyset$, $A_\alpha\cap A_\beta=\emptyset$, contrary to hypothesis. We conclude that either all $A_\alpha\subset B$ (so $C =\emptyset$) or vice-versa.
(e) Let $\{A_\alpha\}_{\alpha\in I}$ be a labeling of all connected subsets containing $x$. Clearly, $A_\alpha\cap A_\beta\neq\emptyset$ since it contains $x$. Thus, $A\equiv\cup A_\alpha$ is connected by (d) and is
\end{document}
% --- Page 67 ---
% LaTeX generation failed because strict JSON was not returned.
% --- Page 68 ---
\documentclass{article}
\usepackage{amsmath}
\usepackage{amssymb}
\usepackage[utf8]{inputenc}
\begin{document}

2.1. Lots of Definitions 47

\noindent\textbf{Proof.} Let $x \in X$ and $A = \{y \in X \mid \text{there is a curve } \gamma \text{ from } x \text{ to } y\}$.
Obviously, $A$ is nonempty. If we prove it is clopen, then $A = X$ and $X$ is
arcwise connected.

Let $y \in A$. By hypothesis, there is an arcwise connected neighborhood
$N$ of $y$. If $z \in N$, we can find a path from $x$ to $y$ and follow it by a path
from $y$ to $z$ and so see that $N \subset A$, that is, $A$ is open.

Let $y \in \bar{A} \setminus A$. Then $y$ is an accumulation point of $A$. So if $N$ is an
arcwise connected neighborhood of $y$, we can find $z \in A \cap N$. Thus, we can
follow a path from $x$ to $z$ by one from $z$ to $y$ and so see $y \in A$ after all.
Thus, $A$ is closed. $\square$

In $\mathbb{R}^n$, $\{x \mid \|x - x_0\| < \rho\}$ is arcwise connected, since for $x, y$ in that ball,
$\gamma(t)=( 1-t)x+ty$ is in the ball. Thus, open sets in $\mathbb{R}^n$ are locally arcwise
connected, and we see

\noindent\textbf{Corollary 2.1.17.} Open connected subsets of $\mathbb{R}^n$ are arcwise connected.

\noindent\textbf{Notes and Historical Remarks.} Topology is a vast subject and includes
areas like combinatorial topology (going back to Euler) and algebraic topol-
ogy (whose father was Poincar´e). Point set topology, our subject in this
chapter, had its roots in the work of Cauchy and Weierstrass, discussed in
the Notes to Section 3.5, and of Cantor and Baire, discussed in the Notes to
Sections 4.2 and 4.3. There were two motivating aspects: First, the attempts
to make sense of surfaces, especially Riemann surfaces, that led to the mod-
erndefinition of manifold. Second, the work of Fredholm [324] on integral
equations, that motivated both Hilbert’s work that led to the theory of
Hilbert spaces, and attempts to discuss convergence on infinite-dimensional
spaces that were concerns of Fr´echet and Riesz.

After several shorter papers, Maurice R´en´e Fr´echet (1878–1973) pub-
lished a comprehensive theory of convergence in his 1906 thesis [315] that
essentially defined axiomatically what we now call metric spaces. Fr´echet
had a weaker condition than the triangle inequality, but mentioned the trian-
gle inequality as a special case. It was F. Riesz (see below) who emphasized
the triangle inequality. The first half of Fr´echet’s thesis also had an attempt
at convergence without a metric. Fr´echet defined Cauchy sequences and, 
without the name, completeness. Most importantly, he was the first to try
to discuss convergence on abstract sets of points rather than $\mathbb{R}^n$ or explicit
spaces of functions.

Between the work of Fr´echet and Hausdorff, F. Riesz [777, 778] devel-
oped some ideas of point set topology and H. Weyl [982] wrote a path-breaking book on Riemann surfaces which emphasized the role of neighbor-
hoods. In 1914, Felix Hausdorff gave birth to the modern theory of point

\noindent\textit{Licensed to AMS.}
\noindent\textit{License or copyright restrictions may apply to redistribution; see http://www.ams.org/publications/ebooks/terms}

\end{document}