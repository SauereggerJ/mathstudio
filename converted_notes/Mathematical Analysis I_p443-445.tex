
% --- Page 443 ---
\documentclass{article}
\usepackage{amsmath, amssymb, amsthm}
\begin{document}

continuous as a function on $\mathbb{R}$, then the new function $(x,y) \mapsto f(\vec{x})$ will be continuous as a function on $\mathbb{R}^2$. This can be verified either directly from the definition
of continuity or by remarking that the function $F$ is the composition $(f \circ \pi_1)(x,y)$
of continuous functions.

In particular, it follows from this, when we take account of c) and e), that the
functions
$$f(x,y) = \sin x + e^{xy}, \quad f(x,y) = \arctan\left( \ln\left( |x|+|y|+1 \right) \right)$$
for example, are continuous on $\mathbb{R}^2$.

We remark that the reasoning just used is essentially local, and the fact that the
functions $f$ and $F$ studied in Example 7 were defined on the entire real line $\mathbb{R}$ or
the plane $\mathbb{R}^2$ respectively was purely an accidental circumstance.

\textbf{Example 8} The function $f(x,y)$ of Example 2 is continuous at any point of the
space $\mathbb{R}^2$ except $(0, 0)$. We remark that, despite the discontinuity of $f(x,y)$ at this
point, the function is continuous in either of its two variables for each fixed value of
the other variable.

\textbf{Example 9} If a function $f : E \to \mathbb{R}^n$ is continuous on the set $E$ and $\tilde{E}$ is a subset
of $E$, then the restriction $f |_{\tilde{E}}$ of $f$ to this subset is continuous on $\tilde{E}$, as follows
immediately from the definition of continuity of a function at a point.

We now turn to the global properties of continuous functions. To state them for
functions $f : E \to \mathbb{R}^n$, we first give two definitions.

\textbf{Definition 8} A mapping $f : E \to \mathbb{R}^n$ of a set $E \subset \mathbb{R}^m$ into $\mathbb{R}^n$ is \textbf{uniformly contin-}
uous on $E$ if for every $\varepsilon> 0$ there is a number $\delta> 0$ such that $d(f(x_1),f(x_2)) < \varepsilon$
for any points $x_1,x_2 \in E$ such that $d(x_1, x_2)<\delta$.

As before, the distances $d(x_1,x_2)$ and $d(f(x_1),f(x_2))$ are assumed to be mea-
sured in $\mathbb{R}^m$ and $\mathbb{R}^n$ respectively.

When $m = n = 1$, this definition is the definition of uniform continuity of
numerical-valued functions that we already know.

\textbf{Definition 9} A set $E \subset \mathbb{R}^m$ is \textbf{pathwise connected} if for any pair of its points $x_0, x_1$,
there exists a path $\Gamma : I \to E$ with support in $E$ and endpoints at these points.
In other words, it is possible to go from any point $x_0 \in E$ to any other point
$x_1 \in E$ without leaving $E$.

Since we shall not be considering any other concept of connectedness for a set
except pathwise connectedness for the time being, for the sake of brevity we shall
temporarily call pathwise connected sets simply \textbf{connected}.

\textbf{Definition 10} A \textbf{domain} in $\mathbb{R}^m$ is an open connected set.

\end{document}
% --- Page 444 ---
\documentclass{article}
\usepackage{amsmath, amssymb, amsfonts}
\usepackage{geometry}
\geometry{a4paper, margin=1in}
\begin{document}
\setcounter{section}{6}
\setcounter{subsection}{0}
\setcounter{equation}{0}
\setcounter{figure}{0}
\setcounter{table}{0}
\setcounter{footnote}{0}
\setcounter{page}{423} % Setting page number close to 424

\section*{7 Functions of Several Variables: Their Limits and Continuity}

\noindent\textbf{Example 10} An open ball $B(\mathbf{a};r)$, $r> 0$, in $\mathbb{R}^m$ is a domain. We already know that $B(\mathbf{a};r)$ is open in $\mathbb{R}^m$. Let us verify that the ball is connected. Let $\mathbf{x}_0 = (x_1^0, \dots, x_m^0)$ and $\mathbf{x}_1 = (x_1^1,\dots, x_m^1)$ be two points of the ball. The path defined by the functions 
$$x_i(t) = t x_i^1 + (1 - t) x_i^0 \quad (i = 1,\dots,m ),$$
defined on the closed interval $0 \le t \le 1$, has $\mathbf{x}_0$ and $\mathbf{x}_1$ as its endpoints. In addition, its support lies in the ball $B(\mathbf{a};r)$, since, by Minkowski’s inequality, for any $t \in [0, 1]$,
$$\begin{aligned}
d(\mathbf{x}(t),\mathbf{a}) &= \sqrt{\sum_{i=1}^{m} (x_i(t) - a_i )^2} \\ &= \sqrt{\sum_{i=1}^{m} \left( t(x_i^1 - a_i) + (1 - t)(x_i^0 - a_i) \right)^2} \\ &\le \sqrt{\sum_{i=1}^{m} (t(x_i^1 - a_i ))^2} + \sqrt{\sum_{i=1}^{m} ((1 - t)(x_i^0 - a_i ))^2} \\ &= t \cdot \sqrt{\sum_{i=1}^{m} (x_i^1 - a_i )^2} + (1 - t) \cdot \sqrt{\sum_{i=1}^{m} (x_i^0 - a_i )^2} \\ &< t r + (1 - t) r = r.
\end{aligned}$$

\noindent\textbf{Example 11} The circle (one-dimensional sphere) of radius $r> 0$ is the subset of $\mathbb{R}^2$ given by the equation $(x_1)^2 + (x_2)^2 = r^2$. Setting $x_1 = r \cos t$, $x_2 = r \sin t$, we see that any two points of the circle can be joined by a path that goes along the circle. Hence a circle is a connected set. However, this set is not a domain in $\mathbb{R}^2$, since it is not open in $\mathbb{R}^2$.

We now state the basic facts about continuous functions in the large.

\subsection*{Global Properties of Continuous Functions}

a) If a mapping $f : K \to \mathbb{R}^n$ is continuous on a compact set $K \subset \mathbb{R}^m$, then it is uniformly continuous on $K$.
b) If a mapping $f : K \to \mathbb{R}^n$ is continuous on a compact set $K \subset \mathbb{R}^m$, then it is bounded on $K$.
c) If a function $f : K \to \mathbb{R}$ is continuous on a compact set $K \subset \mathbb{R}^m$, then it assumes its maximal and minimal values at some points of $K$.
d) If a function $f : E \to \mathbb{R}$ is continuous on a connected set $E$ and assumes the values $f(\mathbf{a}) = A$ and $f(\mathbf{b})= B$ at points $\mathbf{a},\mathbf{b} \in E$, then for any $C$ between $A$ and $B$, there is a point $\mathbf{c} \in E$ at which $f(\mathbf{c})= C$.

Earlier (Sect. 4.2), when we were studying the local and global properties of functions of one variable, we gave proofs of these properties that extend to the more general case considered here. The only change that must be made in the earlier proofs is that expressions of the type $|x_1 - x_2|$ or $|f(\mathbf{x}_1) - f(\mathbf{x}_2)|$ must be replaced by $d(\mathbf{x}_1, \mathbf{x}_2)$ and $d(f(\mathbf{x}_1),f (\mathbf{x}_2))$, where $d$ is the metric in the space where the points in question are located. This remark applies fully to everything except the last statement d), whose proof we now give.

\end{document}
% --- Page 445 ---
% LaTeX generation failed because strict JSON was not returned.