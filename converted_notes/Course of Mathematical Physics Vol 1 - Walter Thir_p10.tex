\documentclass{article}
\usepackage{amsmath}
\usepackage{amsfonts}
\usepackage{amssymb}

\begin{document}

Kleine Vokabelsammlung

$v \cdot w = (v|w) = \text{Skalarprodukt}$ 

$[v \wedge w] \text{ Vektorprodukt}$ 

$\text{grad } f = \text{Gradient}$ 

$\text{rot } f = \text{Rotation}$ 

$\text{div } f = \text{Divergenz}$ 

XI

Summenkonvention: außer wenn unsinnig über doppelte Indizes summieren, etwa $L_{ik} x_k = L_{ik} x_k$

$\|v\|$ Länge des Vektors v = $\left( \sum v_i^2 \right)^{1/2} = d(0,v)$

$ds$ Linienelement

$dO$ Oberflächenelement

$dm_q$ = rn-dimensionales Volumelement

$\perp$ senkrecht

$\| \|$ parallel

$\angle$ Winkel

Gruppen

$GL_n$: Gruppe der n x n-Matrizen M mit $\det M \neq 0$

$O_n$: Gruppe der n x n-Matrizen M mit $M M^t = 1$

$SO_n$: $M \in O_n$, $\det M = 1$

$E_n$: Euklidische Gruppe

$S_n$: Gruppe der Permutationen von n Elementen

$U_n$: Gruppe der komplexen n x n-Matrizen mit $M M^* = 1$

Physikalische Bezeichnungen

$m_i$ Masse des i-ten Teilchens

$x_i$ kartesische Koordinaten des i-ten Teilchens

$t = L$ Zeit

$\tau$ Eigenzeit

$q_i$ verallgemeinerte Koordinaten

$p_i$ verallgemeinerte Impulse

$e_i$ elektrische Ladung des i-ten Teilchens

G Gravitationskonstante

c Lichtgeschwindigkeit

$\hbar$ Wirkungsquantum

$F_{\mu \nu}$ elektrischer Feldstärketensor

$\phi$ Gravitationspotentiale

E elektrische Feldstärke

B magnetische Feldstärke (im Vakuum)

$\mathcal{O}$ von der Ordnung

$\gg$ viel größer

\end{document}